% !TEX TS-program = xelatex
%This Latex template conforms to the class specifications of the Economic Modeling Calss.
\documentclass[12pt,a4paper, margin = 2.5cm]{article}
\usepackage[utf8]{inputenc}
\usepackage{fontspec} 
\usepackage{xunicode} 
\setromanfont[Mapping=tex-text]{Times New Roman} % Schriftart Times New Roman
\usepackage{amsmath}
\usepackage{amsfonts}
\usepackage{amssymb}
\usepackage{graphicx}
\usepackage[backend=bibtex, style=authoryear, uniquename=init, firstinits=true, maxbibnames=3, minbibnames=2]{biblatex} 
\usepackage[doublespacing]{setspace}
\let\cite\textcite
%Add link to bib file here
%\bibliography{literature.bib}

  
\author{Paul Sch\"afer \\\ Student ID: 1336356}



\begin{document}
\title{Analytical Tools from Mary S. Morgan: The World in the Model Chap.:3}
\maketitle
%Two purposes of Visualization
%Space can be seen in two ways: conceptional versus cognitive
\section*{Imagining versus Imaging}

\section*{Perceptual versus Conceptual Space}
A visualization can contain perceptual and conceptual elements. Perceptual elements are elements that are gained by simplification and isolation from the real world. Perceptual elements are elements like indifference curves or contract curves, which are imagined by the economist.
\pagebreak

%\printbibliography

\end{document}